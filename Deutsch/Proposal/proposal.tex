\documentclass[10pt]{book}

\usepackage{import}

% Searches packages and includes within docstyleDE relative to the given path "../"
\subimport{../}{docstyleDE}
% Important: the files docstyleDE.tex and ifiseries.sty must be in the same directory.

\usepackage{pgfgantt}
\ganttset{%
  calendar week text={%
    \currentweek%
  }%
}

\usepackage{rotating}
%\usepackage{lscape}
\usepackage{pdflscape}

\hypersetup{pdftitle=Meine Fake Bachelorarbeit mit einem sehr sehr sehr lange Titel}
\hypersetup{pdfauthor=John Q. Pregraduate}
\hypersetup{pdfsubject=Proposal einer Bachelorarbeit}
\hypersetup{pdfkeywords=}

\begin{document}
\frontmatter
  \thesistitlepage
    {Meine Fake Bachelorarbeit \\[.1em]mit einem sehr sehr \\[.1em]sehr langen Titel}% Title
    {Proposal einer Bachelorarbeit}% Thesistype
    {Hans Eduard Wurst}% Name
    {\today}% Date
    %[..] andere akademische Grade (Bakk., PhD, Bachelor- und Mastergrade mit englischen Bezeichnungen) sind nachzustellen (§ 88 Abs. 2 Universitätsgesetz)
    {M.Sc. weiterer Betreuer} % additional advisors (add more than one with: name1 \\& name2 \\& name3

  \tableofcontents{}
  %\listoffigures{}\listoftables{}\lstlistoflistings{}
  %\chapter*{List of Acronyms}
  %% List of acronyms

\mainmatter

%% the actual thesis
%% use \include{filename} and \includeonly{filename} to organize your thesis

\chapter{Einleitung}
  Dies ist die Einleitung
  \citet{Shaw2003} haben ein ganz tolles Papier geschrieben
  Dies war ein Beispiel für den \textit{natbib} Befehl für \texttt{\textbackslash{}citet\{\}}.
  
  \textit{AspectJ} ist ein tolles Tool~\citep{AspectJ}. Dies war ein Beispiel für den \textit{natbib} Befehl \texttt{\textbackslash{}citep\{\}}.
  
  Kieker: \citep{Rohr2008, Hoorn2009, Hoorn2012}
  
  Artikel Beispiel: \citep{Frey2011}
  
  Wir werden nun die Verwendung von Bildern veranschaulichen (siehe \autoref{fig:figure}).
  
  \begin{figure}[t]%
    \centering%
    \includegraphics[width=0.3\textwidth]{img/template_circle.pdf}%
    \caption{Ein Kreis}%
    \label{fig:figure}%
  \end{figure}%

  Weitere Hinweise auf \url{http://se.informatik.uni-kiel.de/research/scientific-work/} und \url{http://www.twenzel.de/}.

  \section{Motivation}
    \blindtext
  \section{Aufbau}
    \autoref{chp:Foundations} präsentiert alle schönen Grundlagen. \TODO{Anpassen!}

\chapter{Ziele}\label{chp:Goals}
Inhalt:
\begin{itemize}
	\item \textbf{Was} wird angestrebt? => Beschreibung des Problems
	\item \textbf{Warum} wird es angestrebt? => Kontext und Zweck
\end{itemize}

\section{Z1: xxx}
\blindtext

\section{Z2: xxx}
\blindtext

\chapter{Grundlagen und Technologien}\label{chp:Foundations}

Jedes Unterkapitel muss eine Begründung enthalten, die angibt, inwiefern das Unterkapitel relevant für die Thesis ist.

  \section{Grundlage oder Technologie 1}
    \blindtext
  \section{Grundlage oder Technologie 2}
    \blindtext
  \section{Grundlage oder Technologie n}
    \blindtext

\chapter{Ansatz Teil 1}\label{chp:Approach1}
  \section{Ansatz Teil 1 Unterkapitel 1}
    \blindtext
  \section{Ansatz Teil 1 Unterkapitel 2}
    \blindtext
  \section{Ansatz Teil 1 Unterkapitel n}
    \blindtext

\chapter{Zeitplan}\label{chp:Schedule}
  Ein erster Zeitplan und ein Gantt-Chart.
  \section{AP1: Literature Research}
  	\blindtext
  \section{AP2: xxx}
	\blindtext
  \section{AP3: xxx}
    \blindtext
  \section{AP4: Thesis}
    \blindtext
  \section{AP5: Presentation}
	\blindtext

  \section{Gantt Chart}
  \begin{landscape}

	\begin{figure}
		\centering
		\begin{ganttchart}[%Specs
			vgrid,
			hgrid,
	        x unit=0.75mm,
	        y unit chart=0.7cm,
	        link bulge=2.5,
	        milestone right shift=3,
			time slot format=isodate
		]{2015-04-23}{2015-10-31}
		  \gantttitlecalendar{month=shortname, week} \\
		  
		  %\ganttgroup{Preparation}{2015-04-23}{2015-05-31} \\
		  %\ganttgroup{Development}{2015-06-01}{2015-09-14} \\
		  %\ganttgroup{Writing}{2015-04-23}{2015-10-31} \\
		  
		  \ganttbar[name=lit]{AP1: Literature Research}{2015-04-23}{2015-05-25} \\
		  \ganttbar[name=fam]{AP2: xxx}{2015-04-23}{2015-05-31} \\
		  \ganttbar[name=des]{AP3: xxx}{2015-06-01}{2015-06-31} \\
		  %\ganttbar[name=imp]{Implementation}{2015-07-01}{2015-08-31} \\
		  %\ganttbar[name=tes]{Testing}{2015-07-01}{2015-09-14} \\
		  %\ganttbar[name=pro]{Proposal}{2015-04-23}{2015-05-15} \\
		  \ganttbar[name=the]{AP4: Thesis}{2015-05-16}{2015-10-31} \\
		  \ganttbar[name=pre]{AP5: Presentation}{2015-10-01}{2015-10-14} \\
		  
		  \ganttlink{fam}{des}
		  %\ganttlink{des}{imp}
		  %\ganttlink[link mid=0.25]{des}{tes}
		  %\ganttlink{pro}{the}
		\end{ganttchart}
		\caption{Gantt chart showing the schedule of this thesis}
		\label{fig:Gantt}
	 \end{figure}
  \end{landscape}
%%

\backmatter
\tocbibliography

\end{document}
