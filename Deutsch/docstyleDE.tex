\usepackage{etex}		% fix for miktex version 2.9 and above
\reserveinserts{10}		% fix for miktex version 2.9 and above
\usepackage{fixltx2e}
\usepackage[resetfonts]{cmap}
\usepackage{nameref}

%% imports "SE-KCSS-Style"
\usepackage[%
  language=german,paper=a4paper,largepaper=true,%
  algorithm=false,%
  biblatexstyle=authoryear-square,
  biblatexOptions={natbib=true,backend=biber},%
  acronymOptions={smaller,printonlyused},%,withpage
]{ifiseries}

\usepackage{subcaption}
\usepackage{multirow}
\usepackage{rotating}
\usepackage{mathtools}
\usepackage{array}
\usepackage{scrextend}

\usepackage{amsthm}
\theoremstyle{plain}
\newtheorem{thm}{Theorem}[chapter] % reset theorem numbering for each chapter

\theoremstyle{definition}
\newtheorem{defn}[thm]{Definition} % definition numbers are dependent on theorem numbers
\newtheorem{exmp}[thm]{Beispiel} % same for example numbers
\newtheorem{bem}[thm]{Bemerkung}
\newtheorem{lemdef}[thm]{Lemma und Definition}
\newtheorem{lem}[thm]{Lemma}
\newtheorem{ann}{Annahme}

\usepackage[bottom]{footmisc}
\usepackage[format=hang]{caption}[2008/08/24]
\usepackage{textcomp}
\usepackage{todonotes}
\newcommand{\thesistitlepage}[5]{\gentitlepage{#1}{#2}{#3\\\Large\vspace{5ex}#4}{\Large\textsc{Christian-Albrechts-Universit\"{a}t zu Kiel\\Institut f\"{u}r Informatik\\ Arbeitsgruppe Scientific Computing }\\\vspace{10ex}\begin{tabular}{rl}Betreut durch: & Prof. Dr. Steffen B\"{o}rm \\ & #5 \\\end{tabular}}}
\ExecuteBibliographyOptions{sortcase=false,autolang=other,backref=true,abbreviate=false}
\ExecuteBibliographyOptions{isbn=false,url=true,doi=false,eprint=false}
\addbibresource{bibliography.bib}

\hypersetup{bookmarksdepth=3}
\hypersetup{bookmarksopen=true}
\hypersetup{bookmarksopenlevel=0}
\hypersetup{bookmarksnumbered=true}

\newcommand{\TODO}[1]{\todo[inline]{#1}}

\newcommand{\norm}[1]{\left\lVert#1\right\rVert}
\newcommand{\R}{\mathbb{R}}
\newcommand{\N}{\mathbb{N}}
\newcommand{\code}[1]{\lstinline!#1!}
\newcommand{\hmat}{$\mathcal{H}$-Matrizen }
\newcommand{\hquad}{$\mathcal{H}^2$-Matrizen }
\newcommand{\Rk}{\textbf{R}k}
\newcommand{\lag}{\mathcal{L}}
\newcommand{\inds}[1]{{\sigma, #1}}
\newcommand{\indt}[1]{{\tau, #1}}
\newcommand{\ttit}[1]{\text{\textit{#1}}}
\newcommand{\pot}[1]{\mathcal{P}(#1)}
\newcommand{\after}{\circ}
\newcommand{\cstyle}[2]{[language=C, label=lst:#1, caption={#2}, numbers=none]}
\newcommand{\cnumberstyle}[2]{[language=C, label=lst:#1, caption={#2}]}
\newcommand{\tcode}[1]{\text{\code{#1}}}
\newcommand{\trans}[1]{#1^T}
\newcommand{\Proc}{\mathfrak{P}}
\newcommand{\vorw}{\nameref{sec:vorw} }
\newcommand{\ruckw}{\nameref{sec:rückw} }
\newcommand{\koppl}{\nameref{sec:kopplung} }
\newcommand{\vorruck}{\hyperref[sec:vorw]{Vorwärts-} und \nameref{sec:rückw} }
% \newcommand{\qed}{\hfill $\Box$}

\makeatletter
\newcommand{\wlabel}[2]{%
  \phantomsection
  #1\def\@currentlabel{\unexpanded{#1}}\label{#2}%
}
\makeatother

\makeatletter
\newsavebox\myboxA
\newsavebox\myboxB
\newlength\mylenA

\newcommand*\xoverline[2][0.75]{%
    \sbox{\myboxA}{$\m@th#2$}%
    \setbox\myboxB\null% Phantom box
    \ht\myboxB=\ht\myboxA%
    \dp\myboxB=\dp\myboxA%
    \wd\myboxB=#1\wd\myboxA% Scale phantom
    \sbox\myboxB{$\m@th\overline{\copy\myboxB}$}%  Overlined phantom
    \setlength\mylenA{\the\wd\myboxA}%   calc width diff
    \addtolength\mylenA{-\the\wd\myboxB}%
    \ifdim\wd\myboxB<\wd\myboxA%
       \rlap{\hskip 0.5\mylenA\usebox\myboxB}{\usebox\myboxA}%
    \else
        \hskip -0.5\mylenA\rlap{\usebox\myboxA}{\hskip 0.5\mylenA\usebox\myboxB}%
    \fi}
\makeatother

\def \haken#1{\underline{#1}{\raise -0.3ex\hbox{\vphantom{$#1$}\vrule height 0.7ex}}}
\def \nullhaken#1{\underline{#1}{\raise -0.3ex\hbox{\vphantom{$#1$}\vrule height 0.7ex}}_{_0}}

\hyphenation{In-ter-po-la-tions-punk-te}
\hyphenation{Spei-cher-platz-inef-fi-zienz}

\renewcommand{\arraystretch}{1.1}

\renewcommand{\labelitemi}{$\bullet$}
\renewcommand{\labelitemii}{$\circ$}
\renewcommand{\labelitemiii}{$\cdot$}

\renewcommand*{\eidesstatt}{%
  \newpage\thispagestyle{empty}~\newpage
  \thispagestyle{plain}
  \hskip 0mm
  \vfill
  \noindent
  \begin{otherlanguage}{ngerman}
	  \textbf{Eidesstattliche Erkl{\"a}rung}\par
	  \bigskip\noindent Hiermit erkl{\"a}re ich an Eides statt, dass
	  ich die vorliegende Arbeit selbst\-st{\"a}n\-dig verfasst und keine
	  anderen als die angegebenen Quellen und Hilfsmittel verwendet habe.\par
	  \bigskip\noindent Kiel, \today
	  \vskip 10mm
	  \hfill\rule{18em}{.3pt}%
  \end{otherlanguage}}

\lstset{
	frame=single,
	breaklines=true,
	numbers=left,
	numbersep=6pt,
	tabsize=3,
	language=C,
}

\endinput
