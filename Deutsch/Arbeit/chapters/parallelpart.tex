\section{Parallelisierung}
\label{sec:parallelpart}
  Im vorherigen Kapitel haben wir die Berechnung der Gravitationskräfte durch Interpolation approximiert. Dadurch konnten wir die Technik \hquad nutzen um den Rechenaufwand des Algorithmus zu 
  beschleunigen.
  Um die absolute Laufzeit aber noch weiter zu reduzieren sind wir an einer parallel arbeitende Variante dieses Algorithmus interessiert. In diesem Kapitel wird ein Ansatz vorgestellt.
  
  Ziel ist es, zunächst einen einfachen Algorithmus zu entwerfen. Daher beschränken wir uns auf Parallelisierung nach dem Message-Passing-Modell unter Verwendung von MPI. Wir können also 
  $p \in \N$ Prozesse starten. Jeder von diesen hat eine eindeutige $id \in \nullhaken{p-1}$ und seinen privaten Speicherbereich. Insbesondere ist in diesem Modell Shared-Memory ausgeschlossen. 
  Außerdem können die Prozesse miteinander kommunizieren um Daten auszutauschen. Die Menge der Prozesse wird im Folgen mit $\Proc$ bezeichnet.

  
  \subsection{Arbeitsverteilung}
  \label{sec:work}
    Damit wir von parallel arbeitenden Prozessen profitieren können, muss die Arbeit möglichst gleichmäßig auf diese Prozesse verteilt werden. Wir folgen, in modifizierter Variante, dem von 
    \citet{distrh2} vorgestellten Cluster-zentrierten Ansatz. 
    Dieser basiert auf dem Grundgedanken, dass für Vorwärts- und Rückwärtstranformation ausschließlich Daten zwischen Vater- und Sohnclustern ausgetauscht werden müssen. Daher ist es besonders 
    effektiv, wenn möglichst viele Söhne durch den selben Prozess verarbeitet werden, wie der Vater. Besonders einfach wird das Verteilen der Cluster und das Loadbalancing, wenn wir $p$ als 
    Zweierpotenz $p = 2^q$ wählen. Da dann die Anzahl von Clustern in $T_\Omega^{(q)}$ gerade $p$ entspricht, können wir diese Ebene, zuzüglich der Sohncluster, optimal auf die Prozesse verteilen.
    Wir gehen im Folgenden immer von einer so gewählten Anzahl Prozesse aus.
    
    Um diesen Ansatz auf den Algorithmus zu übertragen, wird in der in \autoref{lst:init} aufgeführten init-Methode die globale Variable $\tcode{SPLIT_DEPTH} = log_2(p)$ gesetzt. 
    Außerdem bekommt die Datenstruktur \code Cluster einen weiteren Member: \code{int activ}. In diesem  wird die $id$ des für dieses Cluster zuständigen Prozesses gespeichert.
    
    Zudem gibt es aber noch Cluster auf den Ebenen $T_\Omega^{<q} := T_\Omega^{(0)},\dots,T_\Omega^{(q-1)}$. Um ein Cluster $C \in T_\Omega^{<q}$ zu klassifizieren nutzt jeder Prozess $P \in \Proc$
    mit $id_P$ zwei Konstanten. Gilt für alle Nachfahren $\tilde C \in sons*(C)$ $C->activ \neq id_P$  , wird der Member \code {C->activ} auf die \code{int}-Konstante \code inactiv gesetzt. Gibt es aber Nachfahren,
    für die 
    der Prozess selbst zuständig ist, so wird 
    
  
  \subsection{Datenverteilung}
  \label{sec:data}
    \TODO{Eine der ersten und grundlegenden Fragen, die bei parallel arbeitenden Programmen geklärt werden muss, ist, wo welche Daten vorhanden sind. Ein Möglicher Ansatz wäre, dass jeder Prozess
    eine Kopie aller Daten hat.}