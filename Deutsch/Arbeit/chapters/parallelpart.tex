\section{Parallelisierung}
\label{sec:parallelpart}
  Im vorherigen Kapitel haben wir die Berechnung der Gravitationskräfte durch Interpolation approximiert. Dadurch konnten wir die Technik \hquad nutzen um den Rechenaufwand des Algorithmus zu 
  beschleunigen.
  Um die absolute Laufzeit aber noch weiter zu reduzieren sind wir an einer parallel arbeitende Variante dieses Algorithmus interessiert. In diesem Kapitel wird ein Ansatz vorgestellt.
  
  Ziel ist es, zunächst einen einfachen Algorithmus zu entwerfen. Daher beschränken wir uns auf Parallelisierung nach dem Message-Passing-Modell unter Verwendung von MPI. Wir können also 
  $p \in \N$ Prozesse starten. Jeder von diesen hat eine eindeutige $id \in \nullhaken{p-1}$ und seinen privaten Speicherbereich. Insbesondere ist in diesem Modell Shared-Memory ausgeschlossen. 
  Außerdem können die Prozesse miteinander kommunizieren um Daten auszutauschen. Die Menge der Prozesse wird im Folgen mit $\Proc$ bezeichnet.

  
  \subsection{Arbeitsverteilung}
  \label{sec:work}
    Damit wir von parallel arbeitenden Prozessen profitieren können, muss die Arbeit möglichst gleichmäßig auf diese Prozesse verteilt werden. Wir folgen, in modifizierter Variante, dem von 
    \citet{distrh2} vorgestellten Cluster-zentrierten Ansatz. 
    Dieser basiert auf dem Grundgedanken, dass für \vorruck ausschließlich Daten zwischen Vater- und Sohnclustern ausgetauscht werden müssen. Um möglichst viel Kommunikation zu sparen, ist es daher
    besonders effektiv, wenn möglichst viele Söhne durch den selben Prozess verarbeitet werden, wie der Vater. Besonders einfach wird das Verteilen der Cluster und das Loadbalancing, wenn wir $p$ als 
    Zweierpotenz $p = 2^q$ wählen. Da dann die Anzahl von Clustern in $T_\Omega^{(q)}$ gerade $p$ entspricht, können wir diese Ebene, zuzüglich der Sohncluster, optimal auf die Prozesse verteilen.
    Wir gehen im Folgenden immer von einer so gewählten Anzahl Prozesse aus.
    
    Um diesen Ansatz auf den Algorithmus zu übertragen, wird in der in \autoref{lst:init} aufgeführten init-Methode die globale Variable $\tcode{SPLIT_DEPTH} = log_2(p) = q$ gesetzt. 
    Außerdem bekommt die Datenstruktur \code{Cluster} einen weiteren Member: \code{int active}. In diesem  wird die $id$ des für dieses Cluster zuständigen Prozesses gespeichert.
    
    Zudem gibt es aber noch Cluster auf den Ebenen $T_\Omega^{<q} := T_\Omega^{(0)},\dots,T_\Omega^{(q-1)}$. Um ein Cluster $C \in T_\Omega^{<q}$ zu klassifizieren nutzt jeder Prozess $P \in \Proc$
    mit $id_P$ zwei Konstanten. Gilt für alle Nachfahren $\tilde C \in sons*(C)$ $C.\tcode{active} \neq id_P$  , wird der Member $C.\tcode{active}$ auf die \code{int}-Konstante \code{inactive} gesetzt. 
    Gibt es aber Nachfahren $\tilde C \in sons*(C)$ mit $C.\tcode{active} = id_P$, so wird der Member $C.\tcode{active}$ statt dessen auf die \code{int}-Konstante \code{semi_active} gesetzt. Inaktive
    Cluster werden während der \vorruck nicht weiter beachtet. 
    
    \begin{figure}[t]
    \begin{subfigure}{0.9\textwidth}
    \begin{lstlisting}[label=lst:parsetup, caption={Für die Verteilung der Cluster auf die Prozesse angepasste \code{_setup}-Methode.}]
void _setup(Cluster *c, int depth){
  if(depth == SPLIT_DEPTH){
    c->active = ++split_count;
  }

  if(depth < MAX_DEPTH){
    _setup_nonLeafCluster(c, depth);
  }

  if(c->active == semi_active){
    if(c->son[0]->active == inactive && c->son[1]->active == inactive){
      c->active = inactive;
    } else{
      if(c->son[0]->active != world.rank && c->son[0]->active != semi_active &&
	 c->son[1]->active != world.rank && c->son[1]->active != semi_active) {
	c->active = inactive;
      }
    }
  }
}
    \end{lstlisting}
    \end{subfigure}
    \end{figure}
    Um diese Einteilung vorzunehmen, wird vorrangig der Code der Methode \code{_setup(...)} (vgl. \autoref{lst:setup}) angepasst. 
    Zudem bekommt der Konstruktor \code{_new_bound_Cluster} einen neuen Parameter um den Member \code{active} aller Cluster zu initialisieren. Für die Konstruktion der Wurzel ist dieser auf
    \code{semi_active} gesetzt. Danach wird durch die Methode \code{_setup_nonLeafCluster(...)} immer der Wert des Vaters an die Söhne weitergegeben.
    Der angepasste Code der \code{_setup}-Methode ist in \autoref{lst:parsetup} aufgeführt.
    
    Zunächst wird überprüft, ob die \code{SPLIT_DEPTH} erreicht wurde. Ist dies der Fall wird der Member \code{active} auf 
    die $id$ des zuständigen Prozesses gesetzt. Dies geschieht einfach durch abzählen. Als nächstes folgt der bereits aus \autoref{lst:setup} bekannte Aufruf, der das Teilen in Sohncluster, das 
    Sortieren der \code{bodies} und die Rekursion beinhaltet. Der letzte Teil wird beim Abbau der Rekursion durchgeführt. Hier wird für eben die Cluster in $T_\Omega^{(<q)}$ überprüft ob diese 
    auf \code{semi-active} bleiben, oder, falls kein Nachfahre aktiv ist, der Member \code{active} auf \code{inactive} gesetzt wird.
    
    Die Idee der semi-aktiven Cluster beruht darauf, die Gestalt der Spalten- bzw. Zeilenmatrizen $V_\tau$ und $W_\sigma$ auszunutzen. Diese werden für Nicht-Blattcluster durch die Transfermatrizen 
    aus den Söhnen konstruiert (vgl. \autoref{sec:approxf}). Während der \vorw werden für ein solches Cluster $C_{semi} \in T_\Omega^{<q}$ die Ersatzmassen nur aus (semi-)aktiven Söhnen
    errechnet. Da diese Cluster für mehrere Prozesse als semi-aktiv gekennzeichnet sind, werden global betrachtet alle Söhne in der \vorw beachtet. Ist eines dieser Cluster Bestandteil
    eines zulässigen Blockes $b_0 = C_{semi} \times C$ oder $b1 = C \times C_{semi}$, so können die Prozesse ihre Berechnungen untereinander austauschen. Somit werden die Definitionen der Matrizen
    $V_\tau$ und $W_\sigma$ aus \autoref{eq:vtau} beziehungsweise \autoref{eq:wsigma} lediglich auf mehrere Prozesse verteilt und die Summation bei Bedarf aus den Teilergebnissen gebildet.
    
    Unter der Voraussetzung, dass alle notwendigen Information für jeden Prozess vorhanden sind, ist die einzige Anpassung der \koppl, dass sich die Auswertung auf (semi-)aktive
    Targetcluster beschränkt. Auch die \ruckw braucht sich lediglich auf (semi-)aktive Cluster beschränken.
    
    Letztlich verteilt sich die Arbeit durch diesen Ansatz sehr natürlich auf die Prozesse. In \autoref{fig:vertblock} ist dies veranschaulicht. Hier ist dies für einen Prozess $P$ mit $id_P = 1$ 
    und $p = 4$ dargestellt, welche Zuständigkeiten sich für die Blöcke des Blockbaumes aus der Aufteilung der Targetcluster ergeben.
    
    \begin{figure}[b]
      \includegraphics[width=0.77\textwidth]{img/verteilter_blockbaum.png}
      \caption{Für einen Prozess $P$ mit $id_P = 1$ und die Anzahl Prozesse $p = 4$ ist hier die Zuständigkeit des Prozesses $P$ für Blöcke eines Blockbaumes farbig dargestellt.
	       (Quelle: \citet{h2slides})}
      \label{fig:vertblock}
    \end{figure}

    \clearpage
  
  \subsection{Datenverteilung}
  \label{sec:data}
    \begin{figure}[t]
  \begin{subfigure}{0.9\textwidth}
    \begin{lstlisting}[label=lst:parconsttree, caption={Ausschnitt aus der parallelen Konstruktion des Clusterbaumes.}]
Cluster *constructClusterTree(bodies *b){
  [...]
  preSort(root, 0);
  new_bs = new_bodies(new_n);
  alltoall_bodies(my_bs, send_count, send_displ, 
		  new_bs, recv_count, recv_displ);
  del_bodies(my_bs);
  my_bs = new_bs;
  //reset roots bodies*
  root->bodies = my_bs;
  root->n      = my_bs->n;
  [...]
}
    \end{lstlisting}
  \end{subfigure}
\end{figure}

\begin{figure}[t]
  \begin{subfigure}{0.9\textwidth}
    \begin{lstlisting}[label=lst:presort, caption={Diese Methode sortiert die lokalen \code{bodies} nach Prozesszugehörigkeit.}]
void preSort(Cluster *c, int depth){
  if(depth == SPLIT_DEPTH){
    c->active = ++split_count;
    
    //get count and start index of data to send to process #split_count
    send_count[split_count] = c->n;
    send_displ[split_count] = split_count == 0 ? 0 : 
			      send_displ[split_count - 1] + send_count[split_count - 1];
    
    MPI_Gather(&send_count[split_count], 1, MPI_INT, recv_count, 1, MPI_INT, split_count, MPI_COMM_WORLD);
    if(world.rank == c->active){
      new_n = 0;
      for(int i = 0; i < world.size; i++){
	recv_displ[i] = new_n;
	new_n        += recv_count[i];
      }
    }
  } 
  [...] // Sortierung und rekursiver Aufruf
}
    \end{lstlisting}
  \end{subfigure}
\end{figure}

    Eine weitere grundlegenden Frage ist, wo welche Daten vorhanden sind. Ein Möglicher Ansatz wäre, dass jeder Prozess eine Kopie aller Daten hat. Doch selbst wenn das Gravitationsproblem kein all
    zu speicherhungriges ist, würde der Hauptspeicher bei 32 Prozessen auf einem Knoten\footnote{Dies entspricht den Spezifikationen der meisten Knoten des RZ-Clusters der Uni Kiel.}  sehr schnell 
    knapp werden. Außerdem sei an dieser Stelle an die weitere Skalierbarkeit des Problems erinnert (vgl. \autoref{sec:parrech}). Daher müssen die Daten über die Prozesse verteilt werden. 
    
    Da wir im vorigen Kapitel bereits beschrieben haben, wie die Arbeit effektiv verteilt werden kann, ist es nur naheliegend die Daten auf die gleiche Weise zu verteilen. Jeder Prozess soll also 
    genau die, zu seinen aktiven Clustern gehörenden, Teile der \code{bodies}-Struktur. Dies entspricht den hellblau gekennzeichneten Teilen des Clusterbaumes links in \autoref{fig:vertblock}.
    
    Weder bei reellen Daten, noch bei unseren zufällig generierten Testdaten, können wir davon ausgehen, dass diese nach Clustern sortiert vorliegen. Da ferner bei reellen Daten davon auszugehen
    ist, dass diese aus Platzgründen auch verteilt eingelesen werden müssen, generieren in unserem Programm auch alle Prozesse jeweils zufällige Testdaten. Es wäre möglich die Daten so verteilt
    zu belassen, wie sie generiert beziehungsweise eingelesen wurden. Jedoch wäre dann während der \vorruck eine große Menge an Kommunikation notwendig um diese entsprechend der Arbeitsverteilung
    durchzuführen. Daher ist es sinnvoll diese Daten einmal nach Prozesszugehörigkeit auszutauschen.
    
    Diese Kommunikation wird während der Konstruktion des Clusterbaumes vorgenommen. Der zugehörige Quellcode ist in \autoref{lst:parconsttree} und \autoref{lst:presort} aufgeführt.
    
    Die Methode \code{preSort(...)} sortiert die \code{bodies}, indem sie den Clusterbaum bis zur Ebene \code{SPLIT_DEPTH} konstruiert. In den Arrays \code{send_count[]} und \code{send_displ[]}
    merkt sich jeder Prozess, wie viele Elemente seiner \code{bodies} und ab welcher Position er an welchen anderen Prozess zu senden hat. Respektive werden in den Arrays \code{recv_count[]} 
    und \code{recv_displ[]} die Anzahlen und Pufferpositionen für die zu empfangenden Daten gespeichert. Diese werden über die Methode \code{MPI_Gather(...)} (vgl. \autoref{fig:kolkom} zwischen 
    den Prozessen ausgetauscht. So bekommt jeder Prozess von den anderen mitgeteilt, wie viele Elemente er von ihnen gesendet bekommen wird. Die Gesamtgröße der künftigen \code{bodies}-Memberarrays
    wird in \code{new_n} gespeichert.
    
    Der Datenaustausch findet über die Methode \code{MPI_Alltoallv(..)} (vgl. \autoref{fig:kolkom} und \autoref{lst:a2a}) statt. Diese führt mit Hilfe der zuvor konstruierten \code{count}- und 
    \code{displ}-Arrays gerade eine Transformation der ``Prozess-Daten-Matrix'' aus, wodurch jeder Prozess jedem anderen die zugehörigen Daten schickt und respektive von diesem erhält.
    Danach werden noch einige Variablen aktualisiert. 
    
    Schließlich wird der Clusterbaum wie gehabt vollständig bis zur Ebene \code{MAX_DEPTH} konstruiert. Hierfür sind die ausgetauschten Daten kaum notwendig, da die Konstruktion des Clusterbaumes
    durch mittige Teilung der Nicht-Blattcluster durchgeführt wird und zu diesem Zeitpunkt lediglich die Interpolationspunkte erstellt werden. Jedoch werden so gleich die korrekten Startindizes und
    Anzahlen in den Clustern gespeichert.
    
    In \autoref{sec:konstr} wurde der Programmparameter $\tcode{pot}$ vorgestellt, der die Anzahl an Testsonnen bestimmt. Bei der parallelen Variante generiert jeder Prozess gerade $2^{\tcode{pot}}$
    Sonnen. Die globale Menge $\Omega$ hat also $|\Omega| = 2^q 2^\tcode{pot} = 2^{q+\tcode{pot}}$ Elemente. Daher wird die maximale Baumtiefe $\tcode{MAX_DEPTH} = q + \tcode{pot} - \mathfrak{l}$ 
    gesetzt. Dadurch wird die durchschnittliche Blattgröße von $2^\mathfrak{l}$ Elementen beibehalten.
    
  \subsection{Kommunikation}
  \label{sec:komm}
    \begin{figure}[t]
      \includegraphics[width=0.77\textwidth]{img/kommimpl.png}
      \caption{Hier sind einige Implikationen der Datenverteilung visualisiert.\\
% 	       Dunkelgrün schraffiert: Blöcke die Daten von aus der \vorw von $P_1$ benötigen.\\
% 	       Hellgrün schraffiert: Daten, die $P_1$ für die \koppl und \ruckw benötigt.\\
% 	       Nicht schraffiert: Blöcke für die alle Daten vorliegen.\\
	       (Quelle: \citet{h2slides})}
      \label{fig:kommimpl}
    \end{figure}

    Als Resultat der Aufteilung der Daten müssen wir uns nun Gedanken machen, welche Daten für die Auswertungen der Matrizen zusätzlich gebraucht und damit kommuniziert werden müssen.
    
    In \autoref{kommimpl} sind diese Zusammenhänge in Bezug auf einen Prozess $P_1 \in \Proc$ visualisiert. Die dunkel schraffierten Blöcke sind diejenigen, die Daten von $P_1$ benötigen. 
    Entweder Berechnungen aus der \vorw für zulässige oder die Daten von Sonnen für unzulässige Blöcke. Der Prozess $P_1$ muss die Daten aus den zugehörigen Clustern also an die anderen Prozesse 
    übermitteln. Umgekehrt benötigt $P_1$ die Daten zu den hell schraffierten Blöcken für seine \koppl und \ruckw. Dies kann gerade wieder durch eine Transformation der ``Prozess-Daten-Matrix'' 
    und damit über \code{MPI_Alltoallv(...)} bewerkstelligt werden. 
    
    Diesmal liegen die Daten allerdings nicht sequentiell hintereinander, sonder sind über den Clusterbaum und die \code{bodies}- Arrays verteilt. Hinzu kommt, dass für zulässige und unzulässige
    Blöcke unterschiedliche Daten ausgetauscht werden müssen. Für zulässige Blöcke brauchen lediglich die errechneten Ersatzmassen ausgetauscht werden, da die Interpolationspunkte bereits von jedem 
    Prozess berechnet wurden. Für unzulässige Blöcke hingegen müssen alle zugehörigen Daten aus den Memberarrays \code{x}, \code{y}, \code{z} und \code{m}, sowie die Anzahl der zum Cluster gehörenden
    Sonnen übertragen werden.
    
    Diesmal müssen also für die Kommunikation manuell Sende- und Empfangspuffer angelegt werden. Der zugehörige Code ist in der Datei \code{eval.c} zu finden und wird hier lediglich erläutert.
    
    Die Methode \code{void _prep_comm(Cluster *ct, Cluster *cs)} durchläuft ebenso wie die in \autoref{lst:eval} dargestellte Methode \code{_eval(...)} den impliziten Blockbaum und sucht nach 
    zulässigen und unzulässigen Blöcken mit (semi-)aktivem Targetcluster. Pointer zu diesen Source- und Targetclustern werden in Arrays von Vektoren \footnote{Die Implementierung der vector-Struktur
    sowie zugehörige Methoden wurden von https://gist.github.com/EmilHernvall/953968 übernommen und leicht modifiziert und erweitert.} gespeichert. \footnote{Mit Vektoren sind ist in diesem Fall eine 
    Struktur gemeint, die ein Array mit flexibler Länge darstellt.} Dabei wird darauf geachtet, dass die Daten aus (semi-)aktiven Clustern, für die mehrere Prozesse zuständig sind, für alle diese 
    Prozesse vermerkt werden.
    
    Im Anschluss nutzt die Methode \code{void _prep_buffers()} diese gesammelten Daten um die Sende- und Empfangspuffer zu allozieren und mit den zu sendenden Daten zu füllen. Sind diese Puffer bereit,
    kann die Kommunikation durchgeführt werden. Dazu wird die Methode \code{void _communicate()} aufgerufen. Dies führt mit den entsprechenden Puffern und den gesammelten Anzahl- und Positionsdaten
    insgesamt sechs Aufrufe von \code{MPI_Alltoallv(...)} durch. Eine für die Ersatzmassen der Cluster, und fünf für die benötigten \code{bodies}-Daten, da die einzelnen Koordinatenrichtungen, die 
    Anzahlen pro Cluster und die Massen in einzelnen Arrays vorliegen.
    
    Nachdem die Kommunikation vollendet wurde müssen die Daten noch an die richtigen Stellen vermittelt werden. Die Methode \code{void _finalize_comm()} kopiert die empfangenen Ersatzmassen in die 
    Cluster, sie diese benötigen. Dabei werden die Ersatzmassen von Clustern, die sich aus den Ersatzmassen mehrerer Prozesse zusammensetzen summiert (vgl. \autoref{eq:vtau} bzw. \autoref{eq:wsigma}).
    Die empfangenen \code{bodies}-Daten werden nicht kopiert, um Speicherplatz zu sparen. Die zugehörigen Cluster müssen aber angepasst werden. Der Startindex und die 
    Anzahl wird entsprechend der Empfangspuffer gesetzt. Später können so die Daten direkt aus diesem Puffern abgerufen werden. Dies wäre für die Cluster zwar auch möglich. Da hier die Ersatzmassen
    aber bereits alloziert wurden, belegen wir durch das Kopieren keinen zusätzlichen Speicher, und brauchen so später nicht zu unterscheiden wo welche Ersatzmassen zu finden sind.
    
    Nun kann die \koppl wie gehabt durch die Methode \code{_eval(...)} ausgeführt werden. Die einzigen Anpassungen sind das auslassen von Blöcken mit inaktivem Targetcluster und die Unterscheidung,
    ob die \code{bodies}-Daten in der lokalen \code{bodies} Struct oder in den Empfangspuffern zu finden sind.
    
    Abschließend werden die Daten in inaktiven Clustern durch die Methode \code{void _clear_inactive_clusters()} wieder auf $0$ gesetzt, da diese beim nächsten Simulationsschritt nicht überschrieben
    werden würden. (Semi-)aktive Cluster werden während der \vorw neu berechnet und Cluster für die genau ein andere Prozess zuständig ist werden nach der Kommunikation mit den neuen Daten überschrieben.
    So ist sichergestellt, dass keine alten Daten verbleiben und zu fehlerhaften Berechnungen führen.
    