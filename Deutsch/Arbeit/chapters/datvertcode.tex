\begin{figure}[b]
  \begin{subfigure}{0.9\textwidth}
    \begin{lstlisting}[label=lst:parconsttree, caption={Ausschnitt aus der parallelen Konstruktion des Clusterbaumes.}]
Cluster *constructClusterTree(bodies *b){
  [...]
  ////////////////////////////////////////////////////////
  // Sort the bodies; first locally, then globally.
  preSort(root, 0);
  new_bs = new_bodies(new_n);
  alltoall_bodies(my_bs, send_count, send_displ, 
		  new_bs, recv_count, recv_displ);
  del_bodies(my_bs);
  my_bs = new_bs;
  //reset roots bodies*
  root->bodies = my_bs;
  root->n      = my_bs->n;
  
  //clean up
  [...]
}
    \end{lstlisting}
  \end{subfigure}
\end{figure}

\begin{figure}[t]
  \begin{subfigure}{0.9\textwidth}
    \begin{lstlisting}[label=lst:presort, caption={Diese Methode sortiert die lokalen \code{bodies} nach Prozesszugehörigkeit.}]
void preSort(Cluster *c, int depth){
  if(depth == SPLIT_DEPTH){
    c->active = ++split_count;
    
    //get count and start index of data to send to process #split_count
    send_count[split_count] = c->n;
    send_displ[split_count] = split_count == 0 ? 0 : 
			      send_displ[split_count - 1] + send_count[split_count - 1];
    
    //if the current knot is the active one: gather counts and displs from the other knots:
    MPI_Gather(&send_count[split_count], 1, MPI_INT, recv_count, 1, MPI_INT, split_count, MPI_COMM_WORLD);
    if(world.rank == c->active){
      new_n = 0;
      for(int i = 0; i < world.size; i++){
	recv_displ[i] = new_n;
	new_n        += recv_count[i];
      }
    }
  } 
  [...}] // Sortierung und rekursiver Aufruf
}
    \end{lstlisting}
  \end{subfigure}
\end{figure}
