\begin{table}[t]
 \begin{subtable}{0.45\textwidth}
  \begin{tabular}{c c c}
    \ \ \ \ 
    &
    \begin{tabular}{|c|c|}
      \hline
      \#Elemente & Laufzeit [s] \\
      \hline
      $2^{10}$ & 0,008324 \\
      $2^{11}$ & 0,02569 \\
      $2^{12}$ & 0,04406 \\
      $2^{13}$ & 0,1348 \\
      $2^{14}$ & 0,503 \\
      $2^{15}$ & 0,6866 \\
      $2^{16}$ & 1,905 \\
      $2^{17}$ & 5,271 \\
      $2^{18}$ & 6,454 \\
      $2^{19}$ & 17,52 \\
      $2^{20}$ & 46,63 \\
      $2^{21}$ & 55,57 \\
      $2^{22}$ & 154,5 \\
      $2^{23}$ & 413,1 \\
      $2^{24}$ & 476 \\
      $2^{25}$ & 1270 \\
      \hline
    \end{tabular}
    &
    \ \ \ \ 
  \end{tabular}
 \subcaption{Dieser Testlauf wurde mit $p = 1$ durchgeführt und entspricht damit der nicht-parallelen Variante.}
 \end{subtable} \ \ 
 \begin{subtable}{0.45\textwidth}
  \begin{tabular}{c c c}
    \ \ \ \ 
    &
    \begin{tabular}{|c|c|}
      \hline
      \#Elemente & Laufzeit [s] \\
      \hline
      $2^{10}$ & 0,02034 \\
      $2^{11}$ & 0,04135 \\
      $2^{12}$ & 0,1327 \\
      $2^{13}$ & 0,1744 \\
      $2^{14}$ & 0,4502 \\
      $2^{15}$ & 1,87 \\
      $2^{16}$ & 2,235 \\
      $2^{17}$ & 6,181 \\
      $2^{18}$ & 15,54 \\
      $2^{19}$ & 18,61 \\
      $2^{20}$ & 49,45 \\
      $2^{21}$ & 132,111 \\
      $2^{22}$ & 190,8 \\
      & \\
      & \\
      & \\
      \hline
    \end{tabular}
  \end{tabular}
 \subcaption{Dieser Testlauf wurde mit $p = 32$ durchgeführt, genau ein Knoten des Rechenclusters auszulasten.}
 \end{subtable}
\caption{In dieser Tabelle sind die Laufzeitmessungen der ersten beiden Testläufe aufgeführt.}
\label{tab:1-32x}
\end{table}
