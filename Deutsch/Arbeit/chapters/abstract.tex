\chapter*{Zusammenfassung}
  Um gravitationelle Wechselwirkungen in unserer Galaxie berechnen zu können, müssen Gleichungen gelöst werden, die großen, vollbesetzten Matrizen entsprechen. Eine Möglichkeit, derartige Matrizen
  anzunähern, sind \hquad \kern-0.3em. Um diesen Ansatz nutzen zu können, muss die Berechnung der Gravitationskräfte auf Teilgebieten approximiert werden. Dazu kann beispielsweise Lagrange-Interpolation
  genutzt werden. Durch diese Methodik kann die Berechnung der Gravitationskräfte von $n$ Himmelskörpern und $k$ Interpolationspunkten auf jedem Teilgebiet in $\mathcal{O}(kn)$ Zeiteinheiten gelöst 
  werden. Die Approximation durch Interpolation konvergiert für steigendes $k$ exponentiell gegen die korrekte Lösung. Da in der Regel $k \ll n$ gilt, ist dieser Ansatz wesentlich effizienter als das 
  vollbesetzte Problem zu lösen, welches eine Komplexität von $\mathcal{O}(n^2)$ aufweist.
  
  Für die Dimension einer Galaxie wäre aber auch dieser Ansatz nicht ausreichend. Zum einen wird bei etwa $100$ Milliarden Sonnen weiterhin eine riesige Menge Rechenleistung benötigt, zum anderen
  wird genügend Speicher benötigt, um die benötigten Daten vorzuhalten. 
  Daher wird ferner eine Möglichkeit vorgestellt, die Berechnung und die Daten auf ein Rechencluster zu verteilen. Dabei beschränken wir uns auf das Message-Passing-Modell
  und stellen eine Implementierung unter Verwendung der Bibliothek MPI vor.  
  Durch diese Parallelisierung ist für große Probleme die optimale Komplexität $\mathcal{O}(\frac{kn}{p})$ erreichbar.