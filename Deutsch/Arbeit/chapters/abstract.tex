\chapter*{Zusammenfassung}
  Um gravitationelle Wechselwirkungen in unserer Galaxie berechnen zu können, müssen Gleichungen gelöst werden, die großen, vollbesetzten Matrizen entsprechen. Eine Möglichkeit, derartige Matrizen
  anzunähern, sind \hquad.\todo{Lehrzeichen weg!} Als Approximationsmethode kann dazu Interpolation genutzt werden. Durch diese Methodik kann die Berechnung der Gravitationskräfte von $n$ Himmelskörpern und $k$ 
  Interpolationspunkten in $\mathcal{O}(kn)$ Zeiteinheiten gelöst werden. Die Approximation durch Interpolation konvergiert für steigendes $k$ exponentiell gegen die Lösung. 
  Da in der Regel $k \ll n$ gilt, ist dieser Ansatz wesentlich effizienter als das vollbesetzte Problem zu lösen.
  
  Für die Dimension einer Galaxie wäre aber auch dieser Ansatz nicht ausreichend. Daher wird ferner eine Möglichkeit vorgestellt, die Berechnung auf viele Rechner zu verteilen, die durch ein 
  Netzwerk verbunden sind. Dabei beschränken wir uns auf das Message-Passing-Modell und stellen eine Implementierung unter Verwendung der Bibliothek MPI vor.
  
  \TODO{Ergebnis!}