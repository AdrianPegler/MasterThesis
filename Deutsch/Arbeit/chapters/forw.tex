\begin{figure}[t]
  \begin{subfigure}{0.9\textwidth}
    \begin{lstlisting}[language=C, label=lst:forw, caption={Diese Methode dient dem rekursiven Aufruf der Vorwärtstransformation und delegiert die Konstruktion der Ersatzmassen.
							    Dabei wird zwischen Blatt-und Nicht-Blattclustern unterschieden.}]
void forward(Cluster *c){
  if(c->num_sons){
      _forward_nonLeaf(c);
  } else {
      _forward_leaf(c);
  }
}
    \end{lstlisting}
  \end{subfigure}
\end{figure}

\begin{figure}[p]
  \begin{subfigure}{0.9\textwidth}
    \begin{lstlisting}[language=C, label=lst:forwnonleaf, caption={Diese Methode arbeitet auf den Transfermatrizen und ruft daher zuerst rekursiv die Vorwärtstransformation für die Sohncluster auf.}]
void _forward_nonLeaf(Cluster *c){
  int i, j, k, w, wj, son, ison, json, kson, wson, wsonj;
  double sum, l1, l2, l3;

  forward(c->son[0]);	//recursiv call for first son
  forward(c->son[1]);	//recursiv call for second son

  //for all submasses
  for(i = 0; i < INTERPOLATION_POINTS; i++){
    for(j = 0; j < INTERPOLATION_POINTS; j++){
      wj = i * INTERPOLATION_POINTS + j;
      for(k = 0; k < INTERPOLATION_POINTS; k++){
	w = wj * INTERPOLATION_POINTS + k;
	sum = 0.0;
	
	//for all sons
	for(son = 0; son < 2; son++){
	  //for all sons submasses
	  for(ison = 0; ison < INTERPOLATION_POINTS; ison++){
	    l1 = lagrange(c, i, 0, c->son[son]->xs[ison]);
	    for(json = 0; json < INTERPOLATION_POINTS; json++){
	      wsonj = ison * INTERPOLATION_POINTS + json;
	      l2 = l1 * lagrange(c, j, 1, c->son[son]->xs[INTERPOLATION_POINTS + json]);
	      for(kson = 0; kson < INTERPOLATION_POINTS; kson++){
		wson = wsonj * INTERPOLATION_POINTS + kson;
		l3 = l2 * lagrange(c, k, 2, c->son[son]->xs[2 * INTERPOLATION_POINTS + kson]);

		//calculate the substitution mass
		sum += c->son[son]->m[wson] * l3;
	      }
	    }
	  }
	}
	c->m[w] = sum;
      }
    }
  }
}     
    \end{lstlisting}
  \end{subfigure}
\end{figure}

\begin{figure}[p]
  \begin{subfigure}{0.9\textwidth}
    \begin{lstlisting}[language=C, label=lst:forwleaf, caption={Diese Methode erstellt die Clusterbasis des Clusterbaumes.}]
void _forward_leaf(Cluster *c){
  if(0 == c->n){ return; }
  int i, j, k, l, m, w, wj;
  double x, y, z, *mass, l1, l2, l3;
  bodies *b = c->bodies;
  mass = b->m;

  for(i = 0; i < NUM_SUB_MASSES; i++){
    c->m[i] = 0.0;
  }

  //for all linked actual masses
  for(l = 0; l < c->n; l++){
    m = c->start + l;
    x = b->x[m];
    y = b->y[m];
    z = b->z[m];

    //for all interpolation points:
    for(i = 0; i < INTERPOLATION_POINTS; i++){
      l1 = lagrange(c, i, 0, x);
      for(j = 0; j < INTERPOLATION_POINTS; j++){
	wj = i * INTERPOLATION_POINTS + j;
	l2 = l1 * lagrange(c, j, 1, y);
	for(k = 0; k < INTERPOLATION_POINTS; k++){
	  //actual interpolation point:
	  w = wj * INTERPOLATION_POINTS + k;
	  //actual lagrange weight:
	  l3 = l2 * lagrange(c, k, 2, z);

	  //calculate the substitution mass
	  c->m[w] += mass[m] * l3;
	}
      }
    }
  }
}   
    \end{lstlisting}
  \end{subfigure}
\end{figure}