\begin{figure}[t]
\begin{subfigure}{0.9\textwidth}
\begin{lstlisting}[language=C, label=lst:setup, 
		    caption={Dies sind die Kernmethoden der Konstruktion des Clusterbaumes. Die Teilung der bounding box wird hier vorgenommen
		    sowie die Sortierung der Elemente und Konstruktion der Sohncluster gestartet.}]
void _setup(Cluster *c, int depth){
[...]
  if(depth < MAX_DEPTH){		//split as long as condition is not met
    _setup_nonLeafCluster(c, depth);
  }
[...]
}
void _setup_nonLeafCluster(Cluster *c, int depth){
  int dir = _largest_direction(c);		
  int border = _sortIndices(c, dir);	//number of elements in the first son cluster
  double b1[3],a2[3];			//new bounding box points
  for(int d = 0; d < 3; d++){
    if(d == dir){
      b1[d] = c->center[d];
      a2[d] = c->center[d];
    } else {
      b1[d] = c->b[d];
      a2[d] = c->a[d];
    }
  }
  c->num_sons = 2;
  c->son[0] = _new_bound_Cluster(c->start, border, c->bodies, c->a, b1);
  _setup(c->son[0], depth+1);
  c->son[1] = _new_bound_Cluster(c->start + border, c->n - border, c->bodies, a2, c->b);
  _setup(c->son[1], depth+1);
}
\end{lstlisting}
\end{subfigure}
\end{figure}

\begin{figure}[t]
\begin{subfigure}{0.9\textwidth}
\begin{lstlisting}[language=C, label=lst:init, caption={Hier wird der Wert \code{MAX\_DEPTH} für die Abbruchbedingung berechnet.}]
void init(int argc, String* argv){
[...]
switch(argc){
  case 3:
    pot = atoi(argv[2]);
  case 2:
    INTERPOLATION_POINTS = atoi(argv[1]);
    [...]
}
  NUM_SUB_MASSES = INTERPOLATION_POINTS * INTERPOLATION_POINTS * INTERPOLATION_POINTS;
[...]
  int leaf_pot = INTERPOLATION_POINTS - 1 ? ceil(log2(2 * NUM_SUB_MASSES)) : 4;
  MAX_DEPTH = pot - leaf_pot;
[...]
}
\end{lstlisting}
\end{subfigure}
\end{figure}

\begin{figure}[t]
\begin{subfigure}{0.9\textwidth}
\begin{lstlisting}[language=C, label=lst:sort, caption={Diese Methode nimmt die Sortierung der \code{bodies} anhand einer vorgegebenen Koordinatenrichtung vor.}]
int _sortIndices(Cluster *c, int dir){
  if(0 == c->n){
    return 0;
  }
  int j, front, back;
  double *assoc_coord;
  
  //get associated coordinates of bodies
  assoc_coord = _getAssocCoord(c->bodies, dir);
  
  //sort the bodies according to clusters center
  front = 0;
  back  = c->n;
  back -= 1;
  do {
    j = c->start + front;
    
    if(assoc_coord[j] < c->center[dir]){
      front++;
    } else {
      if(assoc_coord[c->start + back] <= c->center[dir]){
	swap_bodies(c->bodies, c->start + front, c->start + back);
      }
      back--;
    }
  } while(front <= back);

  return front; //front resembles the number of elements in the left half
}
\end{lstlisting}
\end{subfigure}
\end{figure}